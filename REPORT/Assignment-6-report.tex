\documentclass {article}
\usepackage {graphicx}
\usepackage [numbers]{natbib}

\begin {document}

\title {AUTOMATION OF SCHOOLS
SELECTION FOR STUDENTS
IN UGANDA}
\author {Group 9 - CSC-EVE}
\maketitle 

\begin{tabular}{||l||c||r||}
\hline
Name & Student No & Reg No\\
\hline
ARAKA STEPHEN GIFT MUKOYA & 216002334 & 16/K/2148/EVE\\
\hline
AGWA DANIEL & 216009498 & 16/U/2872/EVE\\
\hline
OMODING JOHN & 216009128 & 16/U/11013/EVE\\
\hline 
KAMIKAZI LINDA & 216014116 & 16/U/5329/EVE\\
\hline
\end{tabular}


\newpage

\section{ACKNOWLEDGMENTS}
We thank the Almighty God for giving us the grace, guidance and strength throughout our struggle to ensure that we end successfully.\\

We would like to thank the individuals who welcomed our idea and gave us the information we required without any doubts. \\

We express our gratitude to them for the time they offered us during our research period.\\
\newpage
\tableofcontents
\newpage

\section{INTRODUCTION}
In Uganda’s education system, there are three major qualifications one has to attain before making it to the University which are, Primary Leaving Examinations \textbf{(P.L.E)}, Uganda Certificate of education \textbf{(U.C.E)} and Uganda Advanced Certificate of Education \textbf{(U.A.C.E)}. A student studies for seven, four and two years respectively before moving onto the next level. However, one usually has to attend to a nursery school or kindergarten before joining the primary level of education. The nursery level has no qualification attained, which gives some parents reason to make their children skip this level. \\

At the end of every level, students have an opportunity to join another school where they can study the next level after doing final exams at the current level. These examinations are always sat for annually as the trend has been for a number of years. These exams are set and controlled by a body known as Uganda National Examination Board \textbf{(UNEB)}. \cite{article} The systems of examination format and grading used by UNEB was inherited from University of Cambridge Local Examination Syndicate although this has changed over time. The syndicate was responsible for school examinations in the British colonies until 1968 when the East African Examination Council was formed. It was in 1983 that UNEB became active when the P.L.E section was transferred to them.  \\ 

The process of selecting the school to join is quite long and bureaucratic. Above all, this process is done manually by teachers as shown in the Observer newspaper of March 27\textsuperscript{th}, 2017.\cite{article2} One big problem associated with this process is bribery where some parents buy slots in schools for their children who did not perform to the standards required by that particular school. \\

Although the UNEB curriculum is the most widely recognized curriculum in Uganda, it is important to note that there are schools which offer curriculum of foreign countries for example United States of America, Great Britain etc. For this research topic however, we’ll be focusing on the curriculum offered by UNEB. \\

\subsection{Objectives}

The main objectives of this research are:\\

\begin{description}
  \item[$\ast$ AUTOMATION] to modify the registration process by erasing the manual procedures and automating the system to make it as easy as possible.
  \item[$\ast$ EASE OF USE] to make a system that maximizes on user convenience.
  \item[$\ast$ ACCURACY] by use of the computer to calculate cut-off points and award schools, the system should ensure fairness and equal opportunity to every student.
\end{description}

\subsection{Scope}
Development of the system is intended for Uganda National Examination Board (UNEB). Hence the initial system will include the candidate students in Primary level and secondary schools within Uganda.

\subsection{Limitations}

\begin{description}
  \item[$\bullet$] The system will be limited to Uganda only.
  \item[$\bullet$] For starters, the system will cater for Primary Schools only.
\end{description}

\section{Summary of findings}


\subsection{Description of the Process}
We can summarize the current manual application cycle into 4 steps: registration, school selection, examination, and school allocation.\\

\textbf{Registration}: \\

Students belonging to the candidate class have must register with the national examinations body, UNEB every academic year. UNEB distributes the forms used for registration and students are to fill these in with their details, including personal information, and information pertaining their current school. This is done manually.\\

\textbf{School selection}:\\

After their registration, they select their choice of schools that they would love to go to for their next level of education. e.g. For the primary level, they select from a list of all the secondary schools \cite{article3} in the country, a maximum of four schools which they would love to go to. The selection is made in descending order of priority / preference.\\

\textbf{Examination}:\\

At the end of every academic year, the registered candidates are then required to sit for their final examinations. The purpose of this examination is to gauge their qualification for the next level of study. The result of of this examination also determines the school from which the student can pursue their next level of academics.\\

\textbf{School Allocation:}\\
Every secondary school requires a certain level of excellence to be attained by an aspirant student in order to recruit a him/her. This level of excellence is measured in terms of what is called the Cut-Off points.\\ 
From the results obtained during the examination, the Cut-Off points are obtained. \\
The allocation of schools to a student is dependent on 2 factors:\\
1. The school allocated must be among the 4 that were shortlisted by the student during registration.\\
2. The  student's Cut-Off points must be greater or equal to those required by at least one of the shortlisted schools.\\

\section{MAIN REPORT}
By the time of implementation, the system shall be able to fulfill the tasks as explained below:

\subsection{Registration}
The registration of students will be saved in a database, and will contain their personal information and school details. The confirmation of their school choices will be pending, awaiting their examination and calculation of cut-off points from the presiding results.

\subsection{Calculations}
The system automatically calculates the grades obtained by each student and thereafter computes the cut-off points for each student.\\
The cut-off points are weighed in comparison to each of the required points of a student's selection among the 4 schools of choice.\\
If a student qualifies for recruit to a certain school, the system will assign the student to the list of recruits for the school. However, it is important to note that obtaining the required cut-off points does not guarantee a place into a given school, given other factors to consider.\\
An example of these factors is the case where a school has reached its limit of students to recruit. In this case, the student will have to look into their alternative choices among the other 3 schools out of the total of 4.

\subsection{Communication}
Upon completion of the calculation. There is need to share the information (i.e. the schools) with the concerned share holders (i.e. the schools, the students, and their parents). Through the system, this information is to be shared through both online media and offline media.  

\begin{description}
  \item[$\bullet$ ONLINE MEDIA] will include e-mail.
  \item[$\bullet$ OFFLINE MEDIA] will include SMS.
\end{description}

\section{CONCLUSION}
The research will improve the process of registration in schools,by making it as easy as possible.\\
The system also aims at fairness and providing equal opportunity to all by preventing the malpractices that occur from manual administration \cite{article4}.
Through automation, and considering the accuracy that comes with it, this project not only looks to improve the registration and allocation of schools but to also contribute to the improvement of Uganda's education system as a whole.\\


\begin{thebibliography}{10}

\bibitem{article} Uganda National Examinations Board (2016).  
\emph{Historical Background} [article], 
Available:  \texttt{http://uneb.ac.ug/index.php/our-history/}

\bibitem{article2} Christian Basl, The Observer (2017, Mar. 27). 
\emph(Secondary school selection exercise: who's in control?) [article], 
Available: \texttt{http://observer.ug/education/51981-secondary-schools-selection-exercise-who-is-in-control.html}

\bibitem{article3} A directory to Secondary Schools in Uganda.  
\emph{List of Schools} [article], 
Available:  \texttt{http://www.aboutuganda.com/uganda/education/secondary-schools-in-uganda}

\bibitem{article4} The Persistence of Malpractice in Examinations Conducted by Uganda National Examinations Board at PLE, UCE and UACE Levels.  
\emph{Malpractice} [article], 
Available:  \texttt{http://nilepost.co.ug/2018/03/02/ugandans-are-conspirators-in-examination-malpractices/}


\end{thebibliography}
\end{document}